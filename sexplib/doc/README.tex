\documentclass[a4paper]{article}

\usepackage{hevea}

%BEGIN LATEX
\usepackage{natbib}
%END LATEX

\newcommand{\mail}{\mailto{mmottl@janestcapital.com}}
\newcommand{\homeurl}{http://www.janestcapital.com}
\newcommand{\athome}[2]{\ahref{\homeurl/#1}{#2}}
\newcommand{\www}{\athome{}{Markus Mottl}}

\newcommand{\ocaml}{\ahref{http://www.ocaml.org}{OCaml}}

\newcommand{\myhref}[1]{\ahref{#1}{#1}}
\newcommand{\ocsrc}[2]{\athome{ocaml/#1}{#2}}
\newcommand{\myocsrc}[1]{\athome{ocaml/#1}{#1}}

\newcommand{\janeshort}{\ahref{http://www.janestcapital.com} {Jane Street Holding, LLC}}

\newcommand{\trow}[2]{\quad #1 \quad&\quad #2 \quad\\}
\newcommand{\trowl}[2]{\trow{#1}{#2}\hline}

%BEGIN LATEX
\newcommand{\theyear}{\number\year}
%END LATEX

\title{README: library ``Sexplib''}
\author{
  Copyright \quad (C) \quad \theyear \quad \janeshort \quad\\
  Author: Markus Mottl
}
\date{New York, 2007-11-01}

% DOCUMENT
\begin{document}
\maketitle
\section{Directory contents}
\begin{center}
\begin{tabular}{|c|c|}
\hline
\trowl{CHANGES}{History of code changes}
\trowl{COPYRIGHT}{Notes on copyright}
\trow{INSTALL}{Short notes on compiling and}
\trowl{}{installing the library}
\trowl{LICENSE}{``GNU LESSER GENERAL PUBLIC LICENSE''}
\trowl{LICENSE.Tywith}{License of Tywith, from which Sexplib is derived}
\trowl{Makefile}{Top Makefile}
\trowl{OCamlMakefile}{Generic Makefile for OCaml-projects}
\trowl{OMakefile}{Ignore this file}
\trowl{README}{This file}
\trowl{VERSION}{Current version}
\trowl{lib/}{OCaml-library for S-expression conversions}
\trowl{lib\_test/}{Test applications for the Sexplib-library}
\end{tabular}
\end{center}

\section{What is ``Sexplib''?}

This library contains functionality for parsing and pretty-printing
S-expressions.  In addition to that it contains an extremely useful
preprocessing module for Camlp4, which can be used to automatically
generate code from type definitions for efficiently converting
OCaml-values to S-expressions and vice versa.  In combination with the
parsing and pretty-printing functionality this frees users from having to
write their own I/O-routines for datastructures they define.  Possible
errors during automatic conversions from S-expressions to OCaml-values
are reported in a very human-readable way.  Another module in the library
allows you to extract and replace sub-expressions in S-expressions.

\section{How can you use it?}

The API (.mli-files) in the library directory is fully documented.  Module
``Sexp'' contains all I/O-functions for S-expressions, module ``Conv'' helper
functions for converting OCaml-values of standard types to S-expressions.
Module ``Path'' supports sub-expression extraction and substitution.\\
\\
The module \verb=pa_sexp_conv.ml= contains the extensions for the
Camlp4-preprocessor.  It adds the new construct \verb=with sexp=
(and \verb=with sexp_of= and \verb=with of_sexp=, which are implied by
the first).  When using this construct right after a type definition,
function definitions will be generated automatically, which perform
S-expression conversions.\\
\\
E.g.\ given the following type definition:

\begin{verbatim}
  type t = A | B
  with sexp
\end{verbatim}

The above will generate the functions \verb=sexp_of_t= and
\verb=t_of_sexp=.  The preprocessor also supports automatic addition
of conversion functions to signatures.  Just add \verb=with sexp= to
the type in a signature, and the appropriate function signatures will
be generated.\\
\\
See the file ``lib\_test/conv\_test.ml'' for example usage.  It also
demonstrates how to extract and substitute sub-expressions.

\section{Syntax Specification of S-expressions}

\subsection{Lexical conventions of S-expression}

Whitespace, which consists of space, newline, carriage return, horizontal
tab and form feed, is ignored unless within an OCaml-string, where it
is treated according to OCaml-conventions.  The semicolon introduces
comments.  Comments are ignored, and range up to the next newline
character.  The left parenthesis opens a new list, the right parenthesis
closes it again.  Lists can be empty.  The double quote denotes the
beginning and end of a string following the lexical conventions of OCaml
(see OCaml-manual for details).  All characters other than double quotes,
left- and right parentheses, and whitespace are considered part of a
contiguous string.

\subsection{Grammar of S-expressions}

S-expressions are either strings (= atoms) or lists. The lists can
recursively contain further S-expressions or be empty, and must be
balanced, i.e.\ parentheses must match.

\subsection{Examples}

{\samepage
\begin{verbatim}
  this_is_an_atom_123'&^%!  ; this is a comment
  "another atom in an OCaml-string \"string in a string\" \123"

  ; empty list follows below
  ()

  ; a more complex example
  (
    (
      list in a list  ; comment within a list
      (list in a list in a list)
      42 is the answer to all questions
    )
  )
\end{verbatim}
}

\subsection{Conversion of basic OCaml-values}

Basic OCaml-values like the unit-value, integers (in all representations),
floats, strings, and booleans are represented in S-exp syntax in the
same way as in OCaml.  Strings may also appear without quotes if this
does not clash with the lexical conventions for S-expressions.

\subsection{Conversion of OCaml-tuples}

OCaml-tuples are simple lists of values in the same order as in the tuple.
E.g.:

\begin{verbatim}
  (3.14, "foo", "bar bla", 27) <===> (3.14 foo "bar bla" 27)
\end{verbatim}

\subsection{Conversion of OCaml-records}

OCaml-records are represented as lists of pairs in S-expression syntax.
Each pair consists of the name of the record field (first element),
and its value (second element).  E.g.:

\begin{verbatim}
  {
    foo = 3;
    bar = "some string";
  }

  <===>

  (
    (foo 3)
    (bar "some string")
  )
\end{verbatim}

Type specifications of records allow the use of a special type
\verb=sexp_option= which indicates that a record field should be optional.
E.g.:

\begin{verbatim}
  type t =
    {
      x : int option;
      y : int sexp_option;
    }
\end{verbatim}

The type \verb=sexp_option= is equivalent to ordinary options, but is
treated specially by the code generator.  The above would lead to the
following equivalences of values and S-expressions:
\newpage
\begin{verbatim}
  {
    x = Some 1;
    y = Some 2;
  }

  <===>

  (
    (x (some 1))
    (y 2)
  )
\end{verbatim}

\noindent And:

\begin{verbatim}
  {
    x = None;
    y = None;
  }

  <===>

  (
    (x none)
  )
\end{verbatim}

Note how \verb=sexp_option= allows you to leave away record fields
that should default to \verb=None=.  It is also unnecessary (and
actually wrong) now to write down such a value as an option, i.e.\
the \verb=some=-tag must be dropped if the field should be defined.

\subsection{Conversion of sum types}

Constant constructors in sum types are represented as strings.
Constructors with arguments are represented as lists, the first element
being the constructor name, the rest being its arguments.  Constructors
may also be started in lowercase in S-expressions, but will always be
converted to uppercase when converting from OCaml-values.\\
\\
{\samepage
\noindent For example:

\begin{verbatim}
  type t = A | B of int * float * t with sexp

  B (42, 3.14, B (-1, 2.72, A)) <===> (B 42 3.14 (B -1 2.72 A))
\end{verbatim}

\noindent The above example also demonstrates recursion in datastructures.
}

\subsection{Conversion of variant types}

The conversion of polymorphic variants is almost the same as with
sum types.  The notable difference is that variant constructors must
always start with an either lower- or uppercase character, matching
the way it was specified in the type definition.  This is because OCaml
also distinguishes between upper- and lowercase variant constructors.
Note that type specifications containing unions of variant types are
also supported by the S-expression converter.

\subsection{Conversion of OCaml-lists and arrays}

OCaml-lists and arrays are straightforwardly represented as S-expression
lists.

\subsection{Conversion of option types}

The option type is converted like ordinary polymorphic sum types, i.e.:

\begin{verbatim}
  None        <===>  none
  Some value  <===>  (some value)
\end{verbatim}

There is a deprecated version of the syntax in which values of option
type are represented as lists in S-expressions:

\begin{verbatim}
  None        <===>  ()
  Some value  <===>  (value)
\end{verbatim}

Reading of the old-style S-expression syntax for option values is only
supported if the reference \verb=Conv.read_old_option_format= is set to
\verb=true= (currently the default, which may change soon).  A conversion
exception is raised otherwise.  The old format will be written only if
\verb=Conv.write_old_option_format= is true (also currently the default).
Reading of the new format is always supported.

\subsection{Conversion of polymorphic values}

There is nothing special about polymorphic values as long as there are
conversion functions for the type parameters.  E.g.:

\begin{verbatim}
  type 'a t = A | B of 'a with sexp
  type foo = int t with sexp
\end{verbatim}

In the above case the conversion functions will behave as if \verb=foo=
had been defined as a monomorphic version of \verb=t= with \verb='a=
replaced by \verb=int= on the right hand side.\\
\\
If a datastructure is indeed polymorphic, and you want to convert it,
you will have to supply the conversion functions for the type parameters
at runtime.  E.g.\ in the above example, if you wanted to convert a value
of type \verb='a t=, you would have to write something like this:

\begin{verbatim}
  sexp_of_t sexp_of_a v
\end{verbatim}

\noindent where \verb=sexp_of_a=, which may also be named differently in
this particular case, is a function that converts values of type \verb='a=
to an S-expression.  Types with more than one parameter require passing
conversion functions for those parameters in the order of their appearance
on the left hand side of the type definition.

\subsection{Conversion of abstract datatypes}

Of course, if you want to convert an abstract datatype to an S-expression,
you will have to roll your own conversion function, which should produce
values of type \verb=Sexp.t= directly.  If, however, you want to make
use of your abstract type within definitions of other types, make sure
that you call your conversion function appropriately: it should be in the
same scope as the typename, and must be named \verb=sexp_of_{typename}=.

\subsection{Conversion of hashtables}

Hashtables, which are abstract values in OCaml, are represented as
association lists, i.e.\ lists of key-value pairs, e.g.:

\begin{verbatim}
  ((foo 42) (bar 3))
\end{verbatim}

Reading in the above S-expression as hashtable mapping strings to
integers (\verb=(string, int) Hashtbl.t=) will map \verb="foo"= to $42$
and \verb="bar"= to $3$.  Hashtables will be created with a default size
(at time of writing: 1973).  This size can be changed by setting the
reference \verb=Sexplib.Conv.hashtbl_size= to a different value.\\
\\
Note that the order of elements in the list may matter, because
duplicates are kept: bindings will be inserted into the hashtable in
order of appearence.  Therefore, the last binding of a key will be the
``visible'' one, the others are ``hidden''.  See the OCaml-documentation
on hashtables for details.\\
\\
Note, too, that polymorphic equality may not hold between conversions.
You will have to use a function implementing logical equality for that
purpose.

\section{Contact information}

\noindent In the case of bugs, feature requests and similar, you can
contact us here:\\

\hspace{2ex}\mail\\

\noindent Up-to-date information concerning this library should be
available here:\\

\hspace{2ex}\homeurl/ocaml\\

Enjoy!!\\

\end{document}
